% Please do not change the document class
\documentclass{scrartcl}

% Please do not change these packages
\usepackage[hidelinks]{hyperref}
\usepackage[none]{hyphenat}
\usepackage{setspace}
\doublespace

% You may add additional packages here
\usepackage{amsmath}
\usepackage{graphicx} 
\graphicspath{ {Figures/} }

% Please include a clear, concise, and descriptive title
\title{Research Journal}

% Please do not change the subtitle
\subtitle{COMP220}

% Please put your student number in the author field
\author{1507866}

\begin{document}
	
\maketitle

	
\section{Papers:}
\subsection{Design and Application of General Data Structure for Particle System \cite{Huang}}
\begin{itemize}
	\item The main properties of particle are:
	\begin{itemize}
		\item spatial location
		\item velocity
		\item survival  time
		\item shape
		\item size
		\item gravity factor 
	\end{itemize}
	\item Properties change over time 
	\item Particle system should:
		\begin{itemize}
			\item initialise particles
			\item emit particles
			\item render particles
			\item eliminate particles 
		\end{itemize}
\end{itemize}

Paper mostly just gave a model for a particle system, could be helpful for breaking down the particle system into smaller user stories.


\subsection{Illumination for computer generated pictures \cite{Phong}}
\begin{itemize}
	\item Textures and shadows make 3D graphics look more realistic
	\item Three light parts of light:
	\begin{itemize}
		\item Ambient
		\item Specular
		\item Diffuse
	\end{itemize}
\end{itemize}
Related papers: \begin{itemize}
	\item Blinn \cite{Blinn} - says Phong model not good for reflections
\end{itemize}

\subsection{ Particle Systems --- a Technique for Modeling a Class of Fuzzy Objects \cite{Reeves}}
\begin{itemize}
	\item Other papers refer to this as original paper on Particle Systems
	\item Particle effects for dynamic/ fluid objects 
	\item Particles can ``live" and ``die"
	\item Primitive/simpler shapes/models work better
\end{itemize}
Related papers: \begin{itemize}
	\item Pegoraro \cite{Pegoraro}
\end{itemize}

\subsection{Study on algorithm for fireworks simulation based on particle system \cite{Lei}}
\begin{itemize}
	\item Chose paper as a particle effect for fireworks may relate to a fire particle effect
	\item Particle systems used to simulate irregular objects 
	\item Uses lots of maths
	\item Takes wind/other forces on the particle into account
	\begin{itemize}
		\item 
	\end{itemize}
\end{itemize}

\begin{itemize}
	\item Focuses on water simulation instead of fire
	\item Using physics to make particles more realistic 
	\item Use collision detection to make more fluid / respond to environment 	
\end{itemize}

\subsection{Physically-Based Realistic Fire Rendering \cite{Pegoraro}}
\begin{itemize}
	\item 
\end{itemize}
Related papers: \begin{itemize}
	\item Will add papers from this papers previous research section
\end{itemize}

%\subsection{ \cite{}}
%\begin{itemize}
%	\item 
%\end{itemize}
	
\bibliographystyle{ieeetr}
\bibliography{comp220_2}
	
\end{document}
