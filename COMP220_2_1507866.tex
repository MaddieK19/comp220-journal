% Please do not change the document class
\documentclass{scrartcl}

% Please do not change these packages
\usepackage[hidelinks]{hyperref}
\usepackage[none]{hyphenat}
\usepackage{setspace}
\doublespace

% You may add additional packages here
\usepackage{amsmath}
\usepackage{graphicx} 
\graphicspath{ {Figures/} }

% Please include a clear, concise, and descriptive title
\title{COMP220: Research Journal}

% Please do not change the subtitle
\subtitle{COMP220}

% Please put your student number in the author field
\author{1507866}

\begin{document}
	
\maketitle
\abstract{This essay will look at a lighting in computer graphics and using particle effect systems to simulate dynamic objects such as fire. }
	
\section{Introduction}
The intention of this paper is to look at computer graphic lighting and particle effects. For lighting it will look at Phong's model.
For particle effects it will look at particle effect systems in general, being used for fire works and being used to simulate fire.  

\section{Lighting}
Phong originally proposed a lighting model.  Previous computer graphics only used diffuse lighting, Phong expanded this into three components and introduced highlights \cite{Kajiya}. Phong's model split computer graphic lighting into three components which are specular, ambient and diffuse. Lighting models previous to this could take a long time to calculate or did not give realistic results \cite{Phong}.
Blinn refers to Phong's model as a more realistic model and says it improves the appearance of curved objects \cite{Blinn}. However Blinn and Newell suggested the Phong model was not suitable for reflections \cite{BlinnNewell}...

	
\section{Particle Effects}
Reeves first proposed the idea of a particle system in 1983 \cite{Reeves}. A particle system was to be used to represent dynamic or irregular objects such as clouds, smoke and fire \cite{Reeves, Lei, Nishita}. A particle system would create new particles and make older particles ``die". Reeves' particle technique was one of the first to be proposed \cite{Pegoraro}. There are now alternative methods for simulating dynamic and irregular phenomena as Reeves original technique is now over 30 years old \cite{Nishita, Pegoraro}.

Huang \textit{et al} outline what the main properties of a particle and particle effect system are \cite{Huang}.
A particle system should initialise, emit, render and finally eliminate particles. This corresponds with Reeves as he says particle should be ``born" and ``die", both papers say that particles should have limited lifespans
The properties a particle should have are spatial location, velocity, survival  time, shape, size and a gravity factor \cite{Huang}.

\subsection{Particle Effects for Fireworks and Fire Simulation}
Fireworks are a irregular and dynamic objects so are suited to being simulated using a particle system \cite{Lei, Reeves}. There are many other papers that look at using particle effects for fireworks \cite{Lei, Dong, Zhang}. Lei and Wang looked using a particle system to simulate fireworks \cite{Lei}. Like Reeves they mention particles need to have a lifetime and can be used to create dynamic objects. Their method also takes other forces into account such as simulated temperature, wind speed and physics and what effect this would have on the particle\cite{Lei}. Lei and Wang's method does not follow Huang \textit{et al}'s model as all particles are created at the same time. However, the particles are similar to Huang \textit{et al}'s model, they have attributes such as position, velocity and lifespan.

\bigskip
Another use for particle effects is simulating fire \cite{Pegoraro}. There are many ways to simulate fire with computer graphics \cite{BridaultLouchez, Beaudoin, Lamorlette}. Fire is complex to simulate therefore most methods simplify fire often to just the yellow flame component \cite{Pegoraro}.  Pegoraro and Parker use particle effects to simulate flames but researched fire at a molecular level first to make there simulation more realistic. 

\subsection{Particle Effects for Water Simulation}
Water is another dynamic object that can be simulated using particle effects. Shi and Wang used particle effects to simulate a waterfall \cite{Shi}. Like Lei, Shi and Wang used physics to make the particle movement more natural. Collision detection was also used to make the water move more fluidly.


\section{Conclusion}

	
\bibliographystyle{ieeetr}
\bibliography{comp220_2}
	
\end{document}
