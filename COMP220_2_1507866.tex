% Please do not change the document class
\documentclass{scrartcl}

% Please do not change these packages
\usepackage[hidelinks]{hyperref}
\usepackage[none]{hyphenat}
\usepackage{setspace}
\doublespace

% You may add additional packages here
\usepackage{amsmath}
\usepackage{graphicx} 
\graphicspath{ {Figures/} }

% Please include a clear, concise, and descriptive title
\title{Research Journal}

% Please do not change the subtitle
\subtitle{COMP220}

% Please put your student number in the author field
\author{1507866}

\begin{document}
	
\maketitle

	
\section{Introduction}
	
\section{Section}
Huang \textit{et al} outline what the main properties of a particle and particle effect system are. \cite{Huang}

A particle system should initialise, emit, render and finally eliminate particles.

The properties a particle should have are spatial location, velocity, survival  time, shape, size and a gravity factor.

\bigskip

Blinn originally proposed a lighting model, this model split light into three components that were specular, ambient and diffuse. \cite{Blinn}

	
\section{Conclusion}

	
\bibliographystyle{ieeetr}
\bibliography{comp220_2}
	
\end{document}
