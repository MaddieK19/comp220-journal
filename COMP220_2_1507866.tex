% Please do not change the document class
\documentclass{scrartcl}

% Please do not change these packages
\usepackage[hidelinks]{hyperref}
\usepackage[none]{hyphenat}
\usepackage{setspace}
\doublespace

% You may add additional packages here
\usepackage{amsmath}
\usepackage{graphicx} 
\graphicspath{ {Figures/} }

% Please include a clear, concise, and descriptive title
\title{COMP220: Research Journal}

% Please do not change the subtitle
\subtitle{COMP220}

% Please put your student number in the author field
\author{1507866}

\begin{document}
	
\maketitle

	
\section{Introduction}


\section{Lighting}
Phong originally proposed a lighting model.  Previous computer graphics only used diffuse lighting, Phong expanded this into three components and introduced highlights \cite{Kajiya}. Phong's model split computer graphic lighting into three components which are specular, ambient and diffuse. Lighting models previous to this could take a long time to calculate or did not give realistic results \cite{Phong}.
Blinn refers to Phong's model as a more realistic model and says it improves the appearance of curved objects \cite{Blinn}. However Blinn and Newell suggested the Phong model was not suitable for reflections \cite{BlinnNewell}...

	
\section{Particle Effects}
Reeves first proposed the idea of a particle system in 1983 \cite{Reeves}. A particle system was to be used to represent dynamic or irregular objects such as clouds, smoke and fire \cite{Reeves, Lei}. A particle system would create new particles and make older particles ``die".

Huang \textit{et al} outline what the main properties of a particle and particle effect system are \cite{Huang}.
A particle system should initialise, emit, render and finally eliminate particles. This corresponds with Reeves as he says particle should be ``born" and ``die", both papers say that particles should have limited lifespans
The properties a particle should have are spatial location, velocity, survival  time, shape, size and a gravity factor \cite{Huang}.

\subsection{Particle Effect for Fireworks}
Lei and Wang looked using a particle system to simulate fireworks \cite{Lei}. Like Reeves they mention particles need to have a lifetime and can be used to create dynamic objects. Their method also takes other forces into account such as simulated temperature, wind speed and physics and what effect this would have on the particle\cite{Lei}.

There are many other papers that look at using particle effects for fireworks \cite{Dong, Zhang}.

\subsection{Particle Effect for Fire}
\section{Conclusion}

	
\bibliographystyle{ieeetr}
\bibliography{comp220_2}
	
\end{document}
